
\documentclass[manuscript,screen,review]{acmart}
\twocolumn

\AtBeginDocument{%
  \providecommand\BibTeX{{%
    \normalfont B\kern-0.5em{\scshape i\kern-0.25em b}\kern-0.8em\TeX}}}


%%\setcopyright{acmcopyright}
%%\copyrightyear{2023}
%%\acmYear{2023}
\acmDOI{XXXXXXX.XXXXXXX}

%% These commands are for a PROCEEDINGS abstract or paper.
%% \acmConference[Conference acronym 'XX]{Make sure to enter the correct
 %%  conference title from your rights confirmation emai}{June 03--05,
 %%  2018}{Woodstock, NY}
%
%  Uncomment \acmBooktitle if th title of the proceedings is different
%  from ``Proceedings of ...''!
%
%%\acmBooktitle{Woodstock '18: ACM Symposium on Neural Gaze Detection,
 %%June 03--05, 2018, Woodstock, NY} 
%%\acmPrice{15.00}
%%\acmISBN{978-1-4503-XXXX-X/18/06}

\begin{document}

\title{Reproduction JSON Schema Discovery}

\author{Ilnaz Tayebi}
\email{tayebi01@ads.uni-passau.de}
\orcid{1234-5678-9012}
\affiliation{%
  \institution{University of Passau}
  \city{Passau}
  \country{Germany}
}
\renewcommand{\shortauthors}{Tayebi, I.}

\maketitle

\begin{abstract}
This paper presents a report on the project of reproducing the JSON Schema Discovery, an approach for extracting schemas from JSON and Extended JSON document collections. The project was accomplished by using the same source code and data as the original project and dockerizing it for reproducibility. The report summarizes the key results of the original project and discusses the challenges encountered during the reproduction effort. The importance of using a docker file, GitHub repository, Dio, and Zenodo in creating a reproducible package is also explained. The paper concludes by discussing the differences between Repeat, Reproduce, and Replicate and why the project is considered as reproduction.
\end{abstract}
\section*{Introduction}
In the field of computer science, reproducibility engineering is an important aspect that ensures that research results can be validated and reused by other researchers. The main objective of this report is to describe a project I have worked on as part of the Reproducibility Engineering course. I have tried to create a reproducible package for a project named "Json schema discovery" and I will describe the differences between Repeat, Reproduce, and Replicate, explain why my project considers reproduction only, and give some points about Levels of reproducibility and provenance. Additionally, I will explain the concept of Literate Programming and why I have used LaTeX and knitr to show the results of my experiment in this report.

\section*{Project Description}
The project I have worked on is about creating a reproducible package for a pre-existing project named "Json schema discovery." The project has a GitHub repository and an article about it. I used the Docker file for dockerizing the project. The author did not provide the dataset, so I had the challenge of finding the dataset that they used. I sent an email asking for the dataset and the license, and they provided me with both. However, I encountered difficulties finding the compatible version of the packages since it was not mentioned in the project's documentation.

At the last minute of my work, I encountered unexpected unknown errors, and after two days of work, I found out that the owner of the repository made some commits three days ago. Therefore, in my Docker file, after cloning the repository, I had to check out the specific commit.

\section*{Reproduction Effort}
In this section, I will describe my reproduction effort in detail. The project I worked on is considered a reproduction and not a repetition or replication because I used the same source code and the same data as the main project. I had to overcome various challenges such as finding the dataset and the compatible version of the packages. Additionally, I had to make some modifications to the Docker file to check out a specific commit in the repository.

The use of a Docker file and a GitHub repository, as well as Dio and Zenodo, is important in creating a reproducible package. The Docker file helps in creating a consistent environment for running the project, ensuring that the project runs the same way regardless of the environment it is run on. The GitHub repository serves as a version control system, allowing for collaboration and sharing of the code. Dio and Zenodo are platforms for archiving and sharing digital research outputs. These platforms provide permanent and citable access to the research results, allowing other researchers to validate and reuse the results.

\subsection{Literate Programming and LaTeX}

Literate programming is a software development paradigm that emphasizes human-readable documentation as the primary means of describing the structure and design of code. This approach can help to make code more accessible, maintainable, and reproducible. LaTeX is a powerful typesetting system that is commonly used in scientific and technical writing. Its strength lies in its ability to create well-formatted documents that can be easily typeset for both print and electronic formats.

In this project, LaTeX was used to create a reproducible report that documents the steps taken in creating the reproducible package for the "Json schema discovery" project. The report includes a description of the project and the key results, a discussion of the challenges encountered during the reproduction effort, and a summary of why it is important to use a docker file, GitHub repository, Dio, and Zenodo in creating a reproducible package.

\subsection{Why Reproduction is Important}

The ability to reproduce scientific experiments is crucial for the validity and reliability of scientific findings. It is also essential for scientific progress, as it allows researchers to build upon existing knowledge, test hypotheses, and validate theories. Reproducibility is also important for verifying the reliability and accuracy of research results and for avoiding errors or misunderstandings.

In this project, I considered reproduction as the key aspect of reproducibility engineering. This was because the project I was trying to reproduce was already published and available on GitHub. By following the steps outlined in the project's documentation, I was able to recreate the project's environment and results with a high degree of accuracy. This demonstrates the importance of reproducibility in ensuring the validity and reliability of scientific findings.

\subsection{Challenges Encountered}

During the reproduction effort, I encountered several challenges. One of the main challenges was finding the dataset that the original project used. I had to send an email to the author of the project to request the dataset and license. Once I received the dataset, I encountered another challenge in finding the compatible versions of the packages used in the project. This was because the project's documentation did not mention the versions of the packages used.

Finally, I encountered unexpected unknown errors while working on the project. After two days of troubleshooting, I found out that the owner of the repository had made some commits three days ago. To resolve this issue, I had to check out a specific commit in my docker file after cloning the repository.

\section{Conclusion}
In conclusion, reproducibility is a crucial aspect in the field of computer science and research. It ensures that the experiments and results of a project can be repeated and validated by others. In this project, I attempted to create a reproducible package for a JSON Schema Discovery project, which required me to overcome various challenges such as finding the compatible version of packages and the dataset used in the project.

I used a docker file to dockerize the project and a GitHub repository to store the source code and the docker file. I also used Zenodo and Dio to archive the reproducible package and ensure its longevity.
By providing a reproducible package, I have ensured that the results and the methodology of the project can be validated and replicated by others in the future. This helps to increase the trust in the results and ensures that the scientific literature remains accurate and reliable.
In this report, I also discussed the different levels of reproducibility and the importance of having a reproducible package in the field of computer science. By using latex and knitr, I have shown the results of my experiment in an organized and professional manner, making it easier for others to understand and follow my work.

Overall, this project has emphasized the importance of reproducibility in computer science and research, and has provided me with a better understanding of the challenges and benefits of creating a reproducible package.
\end{document}
\endinput
